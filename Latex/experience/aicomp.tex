% chktex-file 8
\employer{AI Compatible}
\location{Remote}
\dates{August 2025 - Current}
\title{\textbf{Software Developer}}
\begin{position}
    \vspace*{-.1in}
    \begin{itemize}
        \item \textbf{Developing an automated privacy policy analysis model, this included:}
        \begin{itemize} 
            \item Utilises Large Language Models for initial data extraction:  Defining the output form using an OpenAPI schema object for consistently returned format for processing.
            \item Improving and refining the Hader et al's methodology [DOI: 10.1007/s00607-024-01331-9] for a generative, and automatable process to gain more nuanced insight into privacy policy and reduce the amount of API calls.
            % Ensure Below Is Correct
            \item Unsupervised semantic equivalence detection model utilises Baysian Inference, Topology, Linear Algebra, NLP techniques and Non-linear Systems analysis. Part of my system derived an analog for Random k Conditional Nearest Neighbours, a recently published technique for the classification of high-dimensional data.
        \end{itemize}
        \item \textbf{Redeveloping full-stack website:}
        \begin{itemize} 
            \item Defining a PDF creation system that composes PDF document directly a set of data/json using the PDF `page description language'. % More precise
            \item Generalising CSSTypes (Typescript) to be able to create more complex, type-safe style objects.
            \item Creating a geometry engine to compose mathematically definable UI styles, including the capability to have seamless non-uniform backgrounds over disconnected, and unconventionally shaped UI components. % More precise
            \item Simulating a toroidal topology for a scroll-bar to define well ordered cyclical UI wheel elements.  % Explain what this means
            \item Dynamic Background Image generator: Efficiently creates tileable background images on the clientside without the need for image processing tools.
            \item Managing databases with Prisma, PostgreSQL, mongoDB.
            \item Handling authentication, payment processing and third-party APIs.
            \item Hit metadata analysis engine, creates interaction based profiles utilising site-entry points, device and location data to profile and analyse site interactions. % More precise
        \end{itemize}
        \item Managing cloud infrastructure, including virtual machines and networking.% Whats the point
    \end{itemize}
\end{position}


% **Regulatory Analysis Engine:**
%  -  Uses Generative AI, the correct and contemporary tool instead of NLP for core data extraction
%  -  Adapts and improves on academic research
%  -  Implements Baysian Inference **core skill for quant jobs**
%  -  Uses novel techniques like Suprathreshold Stochastic Resonance (SSR), Topology
%  -  Unsupervised

% **Web Development**
% - Learnt an entire new tool chain, and languages (React, Typescript, Prisma, PostgreSQL, mongoDB)
% - **Advanced styling manipulation**
%     - *Explain what skills this shows**
%     - Uses non-trivial styling for front-end (the entire web is based of rectangular containers, this uses complex hexagonal geometry):
%         - Had to build a robust, mathematically derived styling engine `https://github.com/emilbowry/AICompWeb/blob/main/client/src/components/hexagons/hexagon-grid/honeycomb/HexagonRow.styles.ts`
%         - Had to directly monkey-patch a chrome bug `https://issues.chromium.org/issues/40918981#c_ts1687982181`
%         - Improves `https://github.com/frenic/csstype` to allow for more complex styling object vocabulary using advanced typing of (Typescript, the brand new programming language I learnt):
%                 - `https://github.com/emilbowry/AICompWeb/blob/main/client/src/utils/styles.types.ts`
%         - Created a dynamic background image generator to programmatically generate complex, "tilable" background textures entirely on the client-side
%         - Simulating a "euclidian torus" topology to allow for infinite scrolling, **apple doesnt do this** `https://uxdesign.cc/how-apple-fooled-users-with-fake-infinite-scroll-03fed32b112d`
%             - *probably a reason for that that i havent worked out but eh*
%             - *unsure if it is strictly a euclidean torus or just the standard donut torus*
%             - CSS Injection to handle high-performance highlighting, prevent re-rendering by modifying css rules and not the elements
% - **Utilities**
%     - PDF generator that implements the `ISO 32000' PDF specification to generate pdfs from java  objects or json
%     - Contextual telemetry analysis, creates interaction based profiles utilising site-entry points, device and location data to profile user events.
% - **Backend Stuff**
%     - Devising and prototyping a HTTP server from scratch using Rust
%     - Learning `express.js`' to create a production ready HTTP server after I re-learnt the low-level fundamentals
%     - Managing cloud infrastructure, including virtual machines and networking. *Needs to be more specific, since this is a common requirement, maybe experiment using `kubernetes'
% - **Other key skills to later implement**
%     - Directly implement a CI/CD pipeline from scratch
%     - kubernetes
%     - break down exact example and test scripts from engine into ipynb to demonstrate the specific techniques, currently it is an un-understandable mess of spagetti code and versions from initial testing
%     - publish front-end code


%%%%%%%%%%%%%%%%%%%%%%%%%%%%%%%%%%%%%%%%%%%%%%%%%%%%%%%%%%%%
% 1.             What does it solve or fix?                %
% 2.          What skills does it demonstrate?             %
% 3.         Does it use show novel techniques?            %
% 4.    Does it use any advanced mathematical tools?       %
% 5.     Does it implement any novel optimisations?        %
%%%%%%%%%%%%%%%%%%%%%%%%%%%%%%%%%%%%%%%%%%%%%%%%%%%%%%%%%%%%