\textbf{Automated Notes Reasoning:} 
A system that makes inferences about my course notes. Using techniques derived from Category Theory and reverse-engineering parts of the `Obsidian Markdown Editor'.

% A system that makes inferences about my course notes, given their titles. Using techniques derived from Category Theory to deriving understand from syntax and structure to verify and generate notes, infer logical gaps, and missing objects.
% The implementation necessitated reverse engineering parts of the `Obsidian Markdown Editor', in order to have a better integration than was offered in the public API and documentation; parser design: Writing a Lexer/Parser for the custom note title grammar and knowledge graph construction: Automating the generation and verification between objects based on synr rules.

% \begin{itemize}
% 	\item Utilised Universal Properties (specifically Categorical Products) to programmatically define the existence of intersectional notes. This allowed the system to strictly enforce schema completeness by identifying missing `product' nodes between disjoint topics.
% 	\item Modelled the relationship between Note Syntax (Grammar) and Semantic Concepts as an Adjunction. This allowed for bidirectional consistency checking: ensuring every valid grammatical title mapped to a logical object, and every logical gap could be described by a generated title. 
% 	\item Utilises the note structure entirely by how they relate to these `Type Functors' (Definitions, Equations, Methods), leading to understanding of the small note category.
% \end{itemize}




%%%%%%%%%%%%%%%%%%%%%%%%%%%%%%%%%%%%%%%%%%%%%%%%%%%%%%%%%%%%
% 1.             What does it solve or fix?                %
% 2.          What skills does it demonstrate?             %
% 3.         Does it use show novel techniques?            %
% 4.    Does it use any advanced mathematical tools?       %
% 5.     Does it implement any novel optimisations?        %
% 6.     Does it require any extra work or changes?        %
%%%%%%%%%%%%%%%%%%%%%%%%%%%%%%%%%%%%%%%%%%%%%%%%%%%%%%%%%%%%

% 1.a Identifies incomplete notes and areas for study
% 1.b Identifies novel connections between subject

% 2.a Automated Reasoning:
% - Advanced math and logic techniques
% - Parser Design: Writing a Lexer/Parser for the custom note title grammar.
% - Knowledge Graph Construction: Automating the creation of edges (relationships) between nodes (notes) based on semantic rules.

% 2.b Obsidian reverse engineering:
% *[Note] this was legal due to Terms of Service, Section "Restrictions", clause iii exception cases*
%  - Involved analyzing minified internal application logic
%  - Internal API Research

% 3. **NO**

% 4. Potentially category theory and graph theory

% 5. **NO**


% 6.a Explore the category theory relation, ensure it is concrete `https://arxiv.org/pdf/1612.09375`
% 6.b Explore the graph theory relation,
% 6.c publish code


%

