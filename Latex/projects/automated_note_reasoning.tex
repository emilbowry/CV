\textbf{Automated Notes Reasoning:} 
A system that makes inferences about my course notes, given a particular grammar for the note titles. By utilising a common grammar of note titles to be able to define relationships, and types of notes to be able to build higher order structures and relationships between topics and individual notes.
It could identify where I had missed notes, graph connections between notes, understand the relationships between concepts.%, and leveraged tools from category and graph theory.

The implementation necessitated reverse engineering parts of the `Obsidian Markdown Editor', in order to have a better integration than was offered in the public API and documentation. 

% Explore the category theory relation, ensure it is concrete `https://arxiv.org/pdf/1612.09375`
% Explore the graph theory relation,
% publish code


%%%%%%%%%%%%%%%%%%%%%%%%%%%%%%%%%%%%%%%%%%%%%%%%%%%%%%%%%%%%
% 1.             What does it solve or fix?                %
% 2.          What skills does it demonstrate?             %
% 3.         Does it use show novel techniques?            %
% 4.    Does it use any advanced mathematical tools?       %
% 5.     Does it implement any novel optimisations?        %
% 6.     Does it require any extra work or changes?        %
%%%%%%%%%%%%%%%%%%%%%%%%%%%%%%%%%%%%%%%%%%%%%%%%%%%%%%%%%%%%

% 1.a Identifies incomplete notes and areas for study
% 1.b Identifies novel connections between subject

% 2.a Automated Reasoning:
% - Advanced math and logic techniques
% - Parser Design: Writing a Lexer/Parser for the custom note title grammar.
% - Knowledge Graph Construction: Automating the creation of edges (relationships) between nodes (notes) based on semantic rules.

% 2.b Obsidian reverse engineering:
% *[Note] this was legal due to Terms of Service, Section "Restrictions", clause iii exception cases*
%  - Involved analyzing minified internal application logic
%  - Internal API Research

% 3. **NO**

% 4. Potentially category theory and graph theory

% 5. **NO**
