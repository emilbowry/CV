\textbf{Automated Notes Reasoning:} 
A knowledge graph inference engine.
\begin{itemize}
    \item Enforced Universal Properties (Categorical Products) to programmatically identify missing intersectional nodes between disjoint topics, automatically generating bridging notes.
    \item Modelled the relationship between Syntax and Semantics as an Adjunction (Galois Connection), guaranteeing that every grammatical title maps to a logical object.
    \item Applied the Yoneda Lemma to classify notes solely via their morphisms to specific `Type Functors' (Definitions, Equations), enabling property inference without content parsing.
\end{itemize}

%%%%%%%%%%%%%%%%%%%%%%%%%%%%%%%%%%%%%%%%%%%%%%%%%%%%%%%%%%%%
% 1.             What does it solve or fix?                %
% 2.          What skills does it demonstrate?             %
% 3.         Does it use show novel techniques?            %
% 4.    Does it use any advanced mathematical tools?       %
% 5.     Does it implement any novel optimisations?        %
% 6.     Does it require any extra work or changes?        %
%%%%%%%%%%%%%%%%%%%%%%%%%%%%%%%%%%%%%%%%%%%%%%%%%%%%%%%%%%%%

% 1.a Identifies incomplete notes and areas for study
% 1.b Identifies novel connections between subject

% 2.a Automated Reasoning:
% - Advanced math and logic techniques
% - Parser Design: Writing a Lexer/Parser for the custom note title grammar.
% - Knowledge Graph Construction: Automating the creation of edges (relationships) between nodes (notes) based on semantic rules.

% 2.b Obsidian reverse engineering:
% *[Note] this was legal due to Terms of Service, Section "Restrictions", clause iii exception cases*
%  - Involved analyzing minified internal application logic
%  - Internal API Research

% 3. **NO**

% 4. Potentially category theory and graph theory

% 5. **NO**


% 6.a Explore the category theory relation, ensure it is concrete `https://arxiv.org/pdf/1612.09375`
% 6.b Explore the graph theory relation,
% 6.c publish code


%

