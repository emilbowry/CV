% Latex/projects/phasor_average_estimator.tex

\par
\textbf{Phasor Average Estimator}: % BE MORE SPECIFIC
An estimator that models each event as a phasor. Each observation adds a new phasor to the model, allowing us to estimate where we expect to see events in the future. Modelling an event as a phasor allows us to:
\begin{itemize}
	\item Be robust to noise and jitter: Interference of the phasors allows this to cancel out in probability
	\item Handle missed signals: The phasors still process regardless of observing a signal or not.
	\item React instantly to periodicity changes: Most estimators would require some evidence accumulation period before it changes period regime, this just spins a new phasor.
	\item Handle false positives: A new phasor adds to the bank, adding a different frequency component, however until repeated signals are observed, this does not counteract the dominant frequencies.
	\item Make reasonable predictions arbitrarily far into the future, given the events we have seen.
	\item Does not require any exogenous and somewhat arbitrary values like damping ratio since the energy can be bound.
	\item Scale-Invariant: Can handle signals in arbitrary time-frames and amplitudes
\end{itemize}
This was built to solve issues with hardware jitter and Inter-Symbol Interference (ISI), and sensor saturation in a Molecular Communication system.
% Developed a scale-invariant, Phasor-based Statistical Model to solve hardware jitter and Inter-Symbol Interference (ISI), in a Molecular Communication system.

%%%%%%%%%%%%%%%%%%%%%%%%%%%%%%%%%%%%%%%%%%%%%%%%%%%%%%%%%%%%
% 1.             What does it solve or fix?                %
% 2.          What skills does it demonstrate?             %
% 3.         Does it use show novel techniques?            %
% 4.    Does it use any advanced mathematical tools?       %
% 5.     Does it implement any novel optimisations?        %
%%%%%%%%%%%%%%%%%%%%%%%%%%%%%%%%%%%%%%%%%%%%%%%%%%%%%%%%%%%%